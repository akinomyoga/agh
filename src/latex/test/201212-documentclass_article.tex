%% -*- coding:sjis -*-
\documentclass{article}

\usepackage{color}
\DefineNamedColor{named}{r}{rgb}{1,0,0}
%% \DefineNamedColor{hoge}{Blue}{rgb}{0,0,1} %% undefined color model

\title{TestDocumentTitle}
\author{TestAuthor}
\author{TestAuthor2}
\begin{document}
\maketitle

\abstract{Hello}
This is abstract.
That is abstract.
It is abstract.
This is abstract.
That is abstract.
It is abstract.
This is abstract.
That is abstract.
It is abstract.
This is abstract.
That is abstract.
It is abstract.

\tableofcontents

\part{Part}
Here, part 1 starts.
Here, part 1 starts.
Here, part 1 starts.
Here, part 1 starts.
Here, part 1 starts.
Here, part 1 starts.
Here, part 1 starts.
Here, part 1 starts.
Here, part 1 starts.
Here, part 1 starts.
\section{Section} section.
\subsection{SubSection} this is subsection.
\subsubsection{SubSubSection} this is subsubsection.
\paragraph{Paragraph} this is paragraph.
\subparagraph{SubParagraph} subparagraph can also be used.

\part*{Part}
Here, part 1 starts.
\section*{Section} section.
\subsection*{SubSection} this is subsection.
\subsubsection*{SubSubSection} this is subsubsection.
\paragraph*{Paragraph} this is paragraph.
\subparagraph*{SubParagraph} subparagraph can also be used.

\textcolor[named]{Blue}{hello}
\textcolor[named]{JungleGreen}{hello}
\fcolorbox[named]{Red}{JungleGreen}{
  Hello world!
  \begin{tabular}{c|c|c}
  a & b & c\\ \hline
  a & b & c\\ \hline
  a & b & c
  \end{tabular}
  Hello world!\\
  Hello world!
}
���͒��� footnote �𖄂ߍ��ގ����o����\footnote{�Ⴆ�΂���ȕ��ɁB}�Ƃ��������䑶�m�ł�����?
\begin{figure}
  \caption{This is Table Environment}
  \begin{tabular}{c|c|c|c}
  �t & �� & �H & �~ \\ \hline
  spring & summer & fall & winter\\
  spring & summer & autumn & winter
  \end{tabular}
\end{figure}

\begin{figure}
  \caption{This is second figure}
  \caption{This is third figure}
  \caption{This is fourth figure}
\end{figure}


\end{document}
