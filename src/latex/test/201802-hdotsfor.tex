\documentclass[11pt]{revtex4-1}
\usepackage{amsmath}
\begin{document}

\begin{align}
  A &= \begin{pmatrix}
    a_{11} & a_{12} & 0 & \ldots & \ldots & \ldots & 0 \\
    a_{21} & a_{22} & a_{23} & 0 & \ldots & \ldots & 0 \\
    0 & a_{32} & a_{33} & a_{34} & 0 & \ldots & 0 \\
    \hdotsfor{7} \\
    \vdots & \vdots & \vdots & \vdots & \vdots & \vdots & \vdots \\
    \hdotsfor{7} \\
      \vdots & \vdots & \vdots & \vdots & \vdots & \vdots & \vdots \\
    \hdotsfor{7} \\
    0 & \ldots & 0 & a_{n-2,n-3} & a_{n-2,n-2} & a_{n-2,n-1} & 0 \\
    0 & \ldots & \ldots & 0 & q_{n-1,n-2} & a_{n-1,n-1} & a_{n-1,n} \\
    0 & \ldots & \ldots & \ldots & 0 & a_{n,n-1} & a_{nn}
  \end{pmatrix} \\
  &= \begin{pmatrix}
    A_{t-I s-J} & \hdotsfor{4} \\
    \hdotsfor{2} & A_{t-s} & \hdotsfor{2} \\
    \dots & A_{t-1} & A_t & A_{t+1} & \dots \\
    \hdotsfor{2} & A_{t+s} & \hdotsfor{2} \\
    \hdotsfor{4} & A_{t+I s+J}
  \end{pmatrix}
\end{align}

\section{cmd:hdotsfor and pkg:amsmath}

%% \usepackage{amsmath} をしなければ使えない。
%% \usepackage{amsmath} すれば array 環境でも使える。

\begin{equation}
  A = \left(\begin{array}{ccccccc}
    A_{t-I s-J} & \hdotsfor{4} \\
    \hdotsfor{2} & A_{t-s} & \hdotsfor{2} \\
    \dots & A_{t-1} & A_t & A_{t+1} & \dots \\
    \hdotsfor{2} & A_{t+s} & \hdotsfor{2} \\
    \hdotsfor{4} & A_{t+I s+J}
  \end{array}\right)
\end{equation}

%% tabular 環境では \hdotsfor は使えるようにならない。

% \begin{tabular}{ccccccc}
%   $A_{t-I s-J}$ & \hdotsfor{4} \\
%   \hdotsfor{2} & $A_{t-s}$ & \hdotsfor{2} \\
%   $\dots$ & $A_{t-1}$ & $A_t$ & $A_{t+1}$ & $\dots$ \\
%   \hdotsfor{2} & $A_{t+s}$ & \hdotsfor{2} \\
%   \hdotsfor{4} & $A_{t+I s+J}$
% \end{tabular}

%% \hdotsfor より前に内容があるとエラーになる
%% \hdotsfor より後の内容は右寄せで表示される。
%% 点線は内容の左余白だけに表示される。点線の縦位置は baseline に同じ。

\begin{equation}
  A = \left(\begin{array}{ccccccc}
%%    A_{t-I s-J} & X \hdotsfor{4} \\ % Error
    A_{t-I s-J} & \hdotsfor{4} XYZ HELLO \\
    A_{t-I s-J} & \multicolumn{4}{r}{E} XYZ HELLO \\
    A_{t-I s-J} & \hdotsfor{4} \\
    \hdotsfor{2} & A_{t-s} & \hdotsfor{2} \\
    \dots & A_{t-1} & A_t & A_{t+1} & \dots \\
    \hdotsfor{2} & A_{t+s} & \hdotsfor{2} \\
    \hdotsfor{4} & A_{t+I s+J}
  \end{array}\right)
\end{equation}

\subsection{hdotsfor and borders}

%% 右側の border は消える。複数本あっても 0 本になる。
%% multicolumn によって新しく設定することはできる。

\begin{equation}
  A = \left(\begin{array}{||c||c||c||c||c||}
    A_{t-I s-J} & \hdotsfor{4} \\
    \hdotsfor{2} & A_{t-s} & \hdotsfor{2} \\
    \dots & A_{t-1} & A_t & A_{t+1} & \dots \\
    \hdotsfor{2} & A_{t+s} & \hdotsfor{2} \\
    \hdotsfor{4} & \multicolumn{1}{|c|}{A_{t+I s+J}}
  \end{array}\right)
\end{equation}

\begin{center}
  \begin{tabular}{||c||c||c||c||}
    A & \multicolumn{2}{c}{test} & D \\ % \multicolumn の左側の罫線は削除されない。
    A & \multicolumn{2}{|c|}{test} & D \\ % \multicolumn の左側の罫線は特に数が増えたりはしない。
    A & \multicolumn{2}{||c||}{test} & D \\ % こうすると一本増える。
    \multicolumn{2}{c}{test} & C & D \\
    \multicolumn{2}{|c|}{test} & C & D \\
    \multicolumn{2}{||c||}{test} & C & D
  \end{tabular}
\end{center}

\subsection{multicolumn additional contents}

%% multicolumn/hdotsfor の追加内容は親の align に関係なく右 align の様だ。
%% また math mode コマンドは使えない。直立体。
%% paragraph mode の ~ は無視される。つまり、paragraph mode でもない。
\begin{equation}
  A = \left(\begin{array}{ccccccc}
    A_{x-1 y-1} & B_{x+1 y-1} & C_{q+2 r-1} & D_{p+3 q+1} & E \\
    A_{t-I s-J} & \hdotsfor{4} XYZ HELLO \\
    A_{t-I s-J} & \multicolumn{4}{r}{E} XYZ HELLO \\
    A_{t-I s-J} & \multicolumn{4}{c}{E} XYZ HELLO \\
    A_{t-I s-J} & \multicolumn{4}{l}{E} XYZ HELLO \\
    %% A_{t-I s-J} & \hdotsfor{4} X_Y^Z \frac12 \\ % Error
    A_{t-I s-J} & \hdotsfor{4} X~Y~Z \\
  \end{array}\right)
\end{equation}

\subsection{double multicolumn}
Can \verb+\multicplumn+ and \verb+\hdotsfor+ be specified twice in the same cell?

\begin{equation}
  A = \left(\begin{array}{cccc}
    X & Y & Z \\
    % \multicolumn{2}{c}{X Y} \multicolumn{2}{c}{Z W} \\ % Error on the next multicolumn
    % \multicolumn{2}{c}{X Y} \hdotsfor{2} \\ % Error on \hdotsfor (says \@multispan)
    % \hdotsfor{2} \multicolumn{2}{c}{Z W} \\ % Error on \multicolumn (says \@multispan)
    % \hdotsfor{2} \hdotsfor{2} \\ % Error on \hdotsfor (says \@multispan)
  \end{array}\right)
\end{equation}

\end{document}
