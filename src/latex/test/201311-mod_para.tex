%% -*- coding: utf-8 -*-
\documentclass{jsarticle}

\begin{document}

\subsection{{\tt\string\mathfoobar} symbols [2013-11-06]}
取り敢えず text を math に変えた物を全て試してみる。
\begin{center}
{Text-mode Commands.}
\newcommand{\cadn}[1]{\texttt{\string#1}& N/A & N/A}
\newcommand{\cad}[1]{\texttt{\string#1}& $#1$ & N/A}
\newcommand{\cadd}[1]{\texttt{\string#1}& $#1$ & #1}
\begin{tabular}{ | lcc | lcc |}
\hline 
Command  & Math & Text & Command  & Math & Text\\
\hline
\cad{\lbrace}       & \cad{\rbrace} \\ 
\cadn{\mathasciicircum}       & \cadn{\mathless} \\ 
\cadn{\mathasciitilde}        & \cadn{\mathordfeminine} \\ 
\cadn{\mathasteriskcentered} & \cadn{\mathordmasculine} \\ 
\cadn{\mathbackslash}        & \cad{\mathparagraph} \\ 
\cadn{\mathbar}              & \cadn{\mathperiodcentered} \\ 
\cadn{\mathbraceleft}        & \cadn{\mathquestiondown} \\ 
\cadn{\mathbraceright}       & \cadn{\mathquotedblleft} \\ 
\cadn{\mathbullet}           & \cadn{\mathquotedblright} \\ 
\cadn{\mathcopyright}        & \cadn{\mathquoteleft} \\ 
\cadn{\mathdagger}           & \cadn{\mathquoteright} \\ 
\cadn{\mathdaggerdbl}        & \cadn{\mathregistered} \\ 
\cad{\mathdollar}           & \cad{\mathsection} \\ 
\cad{\mathellipsis}         & \cad{\mathsterling} \\ 
\cadn{\mathemdash}           & \cadn{\mathtrademark} \\ 
\cadn{\mathendash}           & \cadd{\mathunderscore} \\ 
\cadn{\mathexclamdown}       & \cadn{\mathvisiblespace} \\ 
\cadn{\mathgreater}&&\\
\hline
\end{tabular}
\end{center}

\subsection{test {\string\string} command in math-mode [2013-11-06]}
$\string\string$ (math-mode) and {\string\string} (text-mode)!
\texttt{\string\string} (text-mode tt)
\begin{center}
\string\hello\"a\\
\string\\\"a\\
\string \"ab\\
\string a\"b\\
%%\string{\"a\"b} エラー
\string $$hello$\\
\end{center}

\begin{itemize}
\item \verb+\string+ は math-mode でも text-mode でも使える。
\item \verb+\string+ は次の一単語を文字列として解釈する。(2番目のコマンドは通常通り。)
\item \verb+\string+ は次の「一文字」に作用する物ではない。(\verb+\\+ は二文字ともそのまま出力される。)
\item \verb+\string+ 直後の単語が通常の文字であっても、それを文字列として解釈し、以降は通常に戻る。
\item \verb+\string+ 直後に空白がある場合は、空白は無視される。
\item \verb+\string+ 直後の \{ も文字として解釈する。(引数として \verb+{}+ で囲んだ複数の単語を指定するとエラーになる。)
\end{itemize}

\subsection{{\tt\string\textfoobar} symbols [2013-11-06]}

\begin{center}
{Text-mode Commands.}
\newcommand{\cad}[1]{\texttt{\string#1}& #1 & N/A}
\newcommand{\cadd}[1]{\texttt{\string#1}& #1 & $#1$}
\begin{tabular}{ | lcc | lcc |}
\hline 
Command  & 表示 & Math & Command  & 表示 & Math\\
\hline
\cad{\textasciicircum}       & \cadd{\textless} \\ 
\cad{\textasciitilde}        & \cadd{\textordfeminine} \\ 
\cadd{\textasteriskcentered} & \cadd{\textordmasculine} \\ 
\cadd{\textbackslash}        & \cadd{\textparagraph} \\ 
\cadd{\textbar}              & \cadd{\textperiodcentered} \\ 
\cadd{\textbraceleft}        & \cadd{\textquestiondown} \\ 
\cadd{\textbraceright}       & \cadd{\textquotedblleft} \\ 
\cadd{\textbullet}           & \cadd{\textquotedblright} \\ 
\cadd{\textcopyright}        & \cadd{\textquoteleft} \\ 
\cadd{\textdagger}           & \cadd{\textquoteright} \\ 
\cadd{\textdaggerdbl}        & \cadd{\textregistered} \\ 
\cad{\textdollar}            & \cadd{\textsection} \\ 
\cadd{\textellipsis}         & \cadd{\textsterling} \\ 
\cadd{\textemdash}           & \cadd{\texttrademark} \\ 
\cadd{\textendash}           & \cadd{\textunderscore} \\ 
\cadd{\textexclamdown}       & \cadd{\textvisiblespace} \\ 
\cadd{\textgreater}&&\\
\hline
\end{tabular}
\end{center}



\end{document}
