\documentclass[11pt]{revtex4-1}
\usepackage{amsmath}
\begin{document}

\section{Questions}
\subsection*{Q: Equation with cmd:tag increases the eq counter?}

\begin{align}
  Hello \\
  World \tag{hello} \\
  Text
\end{align}

A: Does not increase.

\subsection*{Q: Is mathematical formula available in Eq counter?}
\begin{align}
  Hello \tag{$\frac12$}
\end{align}

A: Yes

\subsection*{Q: What is the result with multiple tags}
\begin{align}
  %Hello \tag{tagA} \tag{tagB} \\ % Error
  World \notag \tag{tagC} \\
  Text \tag{tagD} \notag \\
  Type \notag \notag \\
  Time
\end{align}

A: Mutiple cmd:tag are error. Multiple cmd:notag allowed. cmd:notag and cmd:tag can coexist. cmd:notag just disable automatic eqno.

\subsection*{Q: What happens when cmd:tag is used outside math mode}

%\tag{world}

A: Error

\subsection*{Q: What happens when cmd:tag is used in inline math}

%$\tag{hello}$

A: Error

\subsection*{Q: Can cmd:tag be used in nested context such as paragraph modes?}
\begin{align}
  Hello \quad\text{text \tag{tagE}} \\
  World \quad\text{text $math\tag{tagF}$} \\
  Check
\end{align}

A: Yes

\subsection*{Q: Can cmd:notag also be used in nested contexts? Is it the same with cmd:nonumber?}
\begin{align}
  Hello \quad\text{text \notag} \\
  World \quad\text{text $math\notag$} \\
  Hello \quad\text{text \notag} \\
  World \quad\text{text $math\notag$}
\end{align}

A: Yes, Yes.

\end{document}
